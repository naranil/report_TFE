\chapter*{Introduction}
\addcontentsline{toc}{chapter}{Introduction}
\section*{UKP and DIPF}
Ubiquitous Knowledge Processing Lab (also UKP Lab) is a research lab in the Department of Computer Science at the Technische Universit\"at Darmstadt founded in 2006 by Prof. Dr. Iryna Gurevych. UKP Lab develops natural language processing techniques for automatically understanding written text and applies them to information management like information retrieval, question answering, and structuring information in Wikis. Its major realization is the Darmstadt Knowledge Processing Software Repository (DKPro) that offers robust, ready to use NLP components which are built on top of IBM’s Unstructured Information Management Architecture (UIMA) as a common and open framework.

UKP has a partnership with the German Institute for International Educational Research (or DIPF, the German acronym) located in Frankfurt am Main. During my internship I worked with Dr. Habernal in the UKP team in DIPF offices. Our team focus on applying text mining and natural language processing tools to problems related to education. 
\section*{The Project Schedule}
As you can see in \cref{sec:schedule}, the project was divided into several work packages and further sub-tasks. At the beginning, I started with reading some literature about argumentation mining and discovering the annotated corpus. Then, I had to use DKPro and create basic features. While gaining experience in programming and NLP, I went on building fancier features related to sentiment analysis and dependencies. Then I had to evaluate the results of models and perform an error analysis.  