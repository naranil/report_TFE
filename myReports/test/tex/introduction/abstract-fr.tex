\newpage
\section*{R\'esum\'e}
Ce travail vise \`a identifier automatiquement des documents persuasifs et argumentatifs dans des contenus g\'en\'er\'es par l'utilisateur, c'est-\`a-dire des forum en ligne et des commentaires d'articles. Nous avons \`a notre disposition un corpus de 990 documents enrichi de m\'etadonn\'ees telles le th\`eme, la source et la date de cr\'eation. Chaque document est class\'e en tant que persuasif (classe P1) ou non persuasif (classe P2).

Nous avons donc un probl\`eme de classification binaire, qui sera une opportunit\'e d'utiliser des techniques de traitement automatique du langage naturel (TAL) pour extraire des connaissances utiles sur le corpus, de construire des descripteurs complexes et de d\'ecouvrir des outils de machine learning. La plupart des t\^aches r\'ealis\'ees, du traitement des donn\'ees jusqu'\`a la classification ont \'et\'e rendues possibles gr\^ace \`a DKPro, un logiciel de TAL d\'evelopp\'e par le laboratoire UKP.

\subsection*{Mots-Cl\'es}

Traitement automatique du langage naturel, fouille de texte, argumentation mining, feature engineering, machine learning, SVM