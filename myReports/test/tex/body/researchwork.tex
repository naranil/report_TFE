\chapter{The Research Work} 
In this part, we'll see how the theoretical knowledge in Argumentation Theory and NLP presented previously were applied to a concrete case study. We'll first see the types of textual data we have, how they were annotated for the supervised learning task and then how features were engineered to perform an automatic classification. We'll also have a brief look at the classifier algorithm we're using to perform our classification. 
\section{Text Corpus}
The data set used for this study was originally composed of 990 text files. They contain forum posts or articles about 6 different domains related to education, that provoke debate in the American society:
\begin{itemize}
  \item \textbf{homeschooling}: It's the education of children outside the formal settings of public or private schools and is usually undertaken directly by parents or tutors. 
  \item \textbf{redshirting}: The practice of postponing entrance into kindergarten of age-eligible children in order to allow extra time for socioemotional, intellectual, or physical growth. 
  \item \textbf{prayer in schools}: Debate about whether or not a public school should allow and allocate time and buildings for religious practices.
  \item \textbf{public vs private schools}: Which kind of school offers the best education.
  \item \textbf{mainstreaming}: In the context of education, it's the practice of educating students with special needs in regular classes during specific time periods based on their skills.
  \item \textbf{single sex education}: The practice of conducting education where male and female students attend separate classes or in separate buildings or schools.  
\end{itemize} 

The meta-information for each text (id, type of post, domain) is given my the name of the file in itself such as follow:
\\
\\
\centerline{\texttt{174\_P2\_artcomment\_homeschooling.txt}}
\\
\\
Later in the internship, I started to use \textit{XMI} files\footnote{The XML Metadata Interchange (XMI) is an Object Management Group (OMG) standard for exchanging metadata information via Extensible Markup Language (XML).} that contain more information about the post or comment in itself such as the author, the annotator comments, the conflicts, ect... (\cref{sec:xmi})

\subsection{Manual Annotation}
As mentioned before, the classification we want to perform is a supervised learning problem since it wants to imitate the human decision on judging if a post is persuasive or not. An annotation guideline was written by Ivan Habernal \cite{ivanguideline} in order for the annotators to understand the task. In this section, we'll discuss about the general ideas of this guideline.
\subsubsection{Sources of the data}
The textual data that will be used for the studies come from to kind of online sources:
\\
\texttt{artcomment: Article Comments, reactions to online articles}
\\
\texttt{forumpost: Forum Posts, posts in online debates}
\subsubsection{Categories in Persuasion}
\textbf{The task}: Distinguish, whether the comment is persuasive regarding the discussed topic. The key question to answer is: \textit{Does the author intend to convince us clearly about his/her attitude or opinion towards the topic?} If the answer is yes, we classify the comment as persuasive. There are two main categories in this task, namely \texttt{P1:Persuasive} and \texttt{P2:Non-persuasive}. The second category is further divided into more categories, that basically cover the various phenomena that may be encountered in the data.

However, It is not necessary to categorize the data exactly into one of the categories under \texttt{P2:Non-persuasive}. For example, a particular text may be both off-topic and out-of-context; in that case, choose either of these categories.
\\
Remember: we are mainly interested in finding the \texttt{P1:Persuasive} documents that represent on-topic texts with intentions to persuade and convince the readers.
\vspace{0.1cm}
\\
\texttt{Quick overview of possible distinct categories:}
\\
\texttt{P1: Persuasive}
\\
\texttt{P2: Non-persuasive}

\texttt{P2.1: Out-of-context or reaction to other comment}

\texttt{P2.3: Off-topic}

\texttt{P2.4: Personal worries}

\texttt{P2.5: Story-sharing without intentions to persuade}

\texttt{P2.7: Impossible to decide about persuasiveness without deep background knowledge}
\\
\vspace{0.15cm}
While the annotation phase, the annotators were charged of determining if a text was in the P1 class and if not they had to specify to which of the seven P2 subclasses it belongs to.

\subsubsection{Examples}
\textbf{Clearly belongs to P1}
\\
\\
\fbox{\begin{minipage}{1\textwidth}
\textbf{\#2013 (forumpost, single-sex-education)} School should be co-ed. Some children are
awkward with others of their own gender. For example, certain girls who are tom-boys might not be comfortable in a room full of girls. The mixed genders are preparing kids for the real world, where things are not segregated.
\end{minipage}}
\\
\\
In \texttt{\#2013}, the author states that schools should not be single-sex, also provides reasons why he/she thinks so.
\\
\\
\textbf{Problematic P1 case}
\\
\\
This section discusses some examples that were marked as P1 by two annotators but as P2 by a third annotator. We will explain, where the third annotator made an error.
\\
\\
\fbox{\begin{minipage}{1\textwidth}
\textbf{\#300 (artcomment, homeschooling)} This is not representative of most homeschoolers. This is a very, very small minority. Let’s compare that to entire schools in the public school system that cater their teaching to make sure their kids pass the standardized tests so they can keep funding, meanwhile the kids can’t understand concepts that aren’t covered on the tests.

I was homeschooled in Texas, where there is no government oversight of homeschooling. I graduated high school at age 16 with 24 college credits under my belt, was accepted into every university I applied to (all the major schools in Texas), and graduated college in three years at age 19 after being on the Dean’s List every semester except one. Neither of my parents has a college degree and would not be deemed ``qualified'' to teach me. Somehow I didn’t just make it, I thrived. Parental involvement works.
\end{minipage}}
\\
\\
The second paragraph in \texttt{\#300} contains a statement ``parental involvement works'', which is clearly in favor of homeschooling.
\\
\\
\textbf{P2.1: Non-persuasive, out-of-context or reaction to other comment}
\\
\\
\fbox{\begin{minipage}{1\textwidth}
\textbf{\#3245 (artcomment, public-private-schools)} Why are they bad, they still pay taxes but don’t use the service so there is more money for the system to use on fixing its issues. Even when everyone is doing everything they can to fix something does not mean it will be fixed.
\end{minipage}}
\\
\\
In \texttt{\#3245}, without any context, we can only roughly guess what the author is writing about.
\\
\\
\textbf{P2.3: Non-persuasive, off-topic}
\\
\\
\fbox{\begin{minipage}{1\textwidth}
\textbf{\#2049 (artcomment, single-sex-education)} Single-sex education. A poem./ Dearest
people, the people/ always arguing and full of hate,/ why oh why should we ever/ turn out this way? Single-sex,/ co-educational, why does it matter?/ Girls, boys, everyone;/ WE CANNOT REMAIN LIKE THIS/ do you hear me?
\end{minipage}}
\\
\\
\textbf{P2.4: Non-persuasive, personal worries}
\\
\\
\fbox{\begin{minipage}{1\textwidth}
\textbf{\#5024 (artcomment, redshirting)} Oh boy. . . oh my little (but very tall) girl. I’ve chosen to put her into a second year of preschool next year (5 days instead of 3) because I feel that's what is right for her. She's a late October baby, but I'm not sure she's ready for kindergarten. But I worry. Will she be the giant of her class every year? Will there be an opportunity to skip her a grade? She's quite bright, but socially still a little awkward. I don't feel I'm ``holding her back", yet if she has brand new twin sisters arriving in July, should I totally turn her world upside down and ship her off to another school with mostly older kids? I'm torn (and totally on the fence) both ways. I want her to excel academically, but I don't want to throw too many changes at her at once. I'm with you, Erica. I'm torn, and I chose the now unpopular ``redshirting", but not so she can be a hockey superstar. . . :) I just thought this was a better pace for her. In ten years, I'm sure the ``experts" will be telling me I should have held
her back, because all the young kids are struggling. . . You can't win.
\end{minipage}}
\\
\\
In \texttt{\#5024}, the author only expresses her worries about her child, but she neither takes stance on the topic nor argues about that.
\\
\\
\textbf{P2.5: Non-persuasive, story-sharing without intentions to persuade}
\\
\\
\fbox{\begin{minipage}{1\textwidth}
\textbf{\#5030 (artcomment, redshirting)} Born in November, my youngest sister was among the oldest children in her peer group until she skipped a grade (I believe she skipped grade one but it may have been grade two). My other sister, also born in November and two years older, showed my youngest sister her homework and my youngest sister proved such a quick learner the teacher had no choice but to recommend she be moved up. She’s still achieving plenty and has never been intimidated by anyone older. She has a competitive drive and enjoys pushing herself forward.
\end{minipage}}
\\
\\
The purpose of \texttt{\#5030} was to share the story without taking stance towards the topic or persuading others (the story of her sister skipping a grade and doing well could is also too far from the redshirting topic).
\\
\\
\textbf{P2.7: Non-persuasive, impossible to decide about persuasiveness without deep back-ground knowledge}
\\
\\
\fbox{\begin{minipage}{1\textwidth}
\textbf{\#164 (artcomment, homeschooling)}: Child abuse in the name of religious freedom. Just like parents who refuse medical treatment for their children. It makes me wish there was a hell.
\end{minipage}}
\\
\\
In \texttt{\#164}, without knowing the context that homeschooling and religion education are somehow related issues in some communities, it is not possible to decide about persuasiveness of this document.

\subsubsection{Annotation Process}
Three annotators were charged of labelling the data as P1 and P2 (if P2, he was asked to specify the subclass). For every annotations, the annotator could write a comment about why he thinks the text can be considered as persuasive or not. This comment can be useful especially if there's a conflict among the 3 annotators. 

Once the annotations are performed, a discussion takes place with the annotators in order to solve issues and conflict annotations. If all annotators agree on the class (P1 or P2) of a text, the class will be set as the \emph{gold label} of this text. But if after the discussion, there's still a conflict, the text will be labelled according to majority. 

To evaluate how well were the annotations, we compare statistical metrics that are described in~\cref{sec:bincla} such as Recall, Precision, Accuracy and Macro $F_{1}$ measure. The comparison will be performed on 4 scenario: 
\begin{itemize}
  \item \textbf{$A_1$ vs $A_2$}
  \item \textbf{$A_1$ vs $A_3$}
  \item \textbf{$A_2$ vs $A_3$}
  \item 3 Annotators \textbf{vs} \Gls{gold data}
\end{itemize}
$A_1$, $A_2$, $A_3$ stand for Annotator number 1, 2 and 3. 

\subsubsection{Results of the Manual Annotations}
We measure the performances of the annotations on 3 batches of text data, and we aggregate the results:
\begin{table}[H]
\begin{tabular}{lrrr|rrr|rrr}
	&	&	&	& \multicolumn{3}{l}{Persuasive} & \multicolumn{3}{l}{Non-Persuasive} \\
	& Docs	& Macro $F_1$ & Acc. & P & R	& $F_1$ & P & R & $F_1$ \\ \hline
Batch 1	&	&	&	&	&	&	&	&	&\\
A1	&100	&0.879	&0.880	&0.942	&0.845	&0.891	&0.813	&0.929	&0.867\\
A2	&100	&0.895	&0.900	&0.875	&0.966	&0.918	&0.944	&0.810	&0.872\\
A3	&100	&0.849	&0.850	&0.922	&0.810	&0.862	&0.776	&0.905	&0.835\\ \hline
Batch 2	&	&	&	&	&	&	&	&	&\\
A1	&200	&0.855	&0.855	&0.909	&0.792	&0.847	&0.813	&0.919	&0.863\\
A2	&200	&0.910	&0.910	&0.919	&0.901	&0.910	&0.901	&0.919	&0.910\\
A3	&200	&0.874	&0.875	&0.839	&0.931	&0.883	&0.920	&0.818	&0.866\\ \hline
Batch 3	&	&	&	&	&	&	&	&	&\\
A1	&509	&0.927	&0.927	&0.953	&0.906	&0.929	&0.902	&0.950	&0.926\\
A2	&502	&0.879	&0.884	&0.836	&0.986	&0.905	&0.977	&0.757	&0.853\\
A3	&511	&0.907	&0.908	&0.977	&0.835	&0.900	&0.857	&0.981	&0.915\\ \hline
All data	&	&	&	&	&	&	&	&	&\\
A1	&809	&0.904	&0.904	&0.942	&0.871	&0.905	&0.867	&0.940	&0.902\\
A2	&802	&0.890	&0.893	&0.858	&0.964	&0.908	&0.948	&0.807	&0.872\\
A3	&811	&0.893	&0.893	&0.929	&0.855	&0.890	&0.861	&0.932	&0.895\\
\end{tabular}
\caption{\label{tab:human-performance-annotation-study-1} Human performance on \emph{gold data persuasive}.}
\end{table}
The Macro F-measure (here called $F_1$) and the accuracy are high enough to consider an automatic classification based on machine learning.

\subsection{Corpus Statistics}
Now that we have the \emph{gold labels} for the texts, we can sum the relevant information in a table: 
\begin{table}[h]
\centering
\begin{tabular}{ll|l|l|l|l|l|l|l|}
\cline{3-9}
 &  & RS & PIS & HS & SSE & MS & PPS & Total \\ \hline
\multicolumn{1}{|l|}{} & artcomment & 24 & 60 & 64 & 17 & 1 & 278 & 444 \\ \cline{2-9} 
\multicolumn{1}{|l|}{} & forumpost & 14 & 17 & 22 & 9 & 9 & 9 & 80 \\ \cline{2-9} 
\multicolumn{1}{|l|}{\multirow{-3}{*}{P1}} & \cellcolor[HTML]{C0C0C0}all & \cellcolor[HTML]{C0C0C0}38 & \cellcolor[HTML]{C0C0C0}77 & \cellcolor[HTML]{C0C0C0}86 & \cellcolor[HTML]{C0C0C0}26 & \cellcolor[HTML]{C0C0C0}10 & \cellcolor[HTML]{C0C0C0}287 & \cellcolor[HTML]{C0C0C0}524 \\ \hline
\multicolumn{1}{|l|}{} & artcomment & 15 & 43 & 93 & 16 & 2 & 174 & 343 \\ \cline{2-9} 
\multicolumn{1}{|l|}{} & forumpost & 15 & 23 & 45 & 8 & 17 & 15 & 123 \\ \cline{2-9} 
\multicolumn{1}{|l|}{\multirow{-3}{*}{P2}} & \cellcolor[HTML]{C0C0C0}all & \cellcolor[HTML]{C0C0C0}30 & \cellcolor[HTML]{C0C0C0}66 & \cellcolor[HTML]{C0C0C0}138 & \cellcolor[HTML]{C0C0C0}24 & \cellcolor[HTML]{C0C0C0}19 & \cellcolor[HTML]{C0C0C0}189 & \cellcolor[HTML]{C0C0C0}466 \\ \hline
\multicolumn{1}{|l|}{} & Total & 68 & 143 & 224 & 50 & 29 & 476 & 990 \\ \cline{2-9} 
\multicolumn{1}{|l|}{\multirow{-2}{*}{P1  + P2}} & Percentage & 6.9 & 14.4 & 22.6 & 5.1 & 2.9 & 48.1 & 100 \\ \hline
\end{tabular}
\end{table}
\\
Abbreviations:

\texttt{RS = redshirting}

\texttt{PIS = prayer in school}

\texttt{HS = homeschooling}

\texttt{SSE = single sex education}

\texttt{MS = mainstreaming}

\texttt{PPS = public private schools}

\subsection{Conflict Annotations}
\label{sec:conflictannotations}
When an annotator doesn't agree with the others about the class of a certain text, a discussion takes place between the three annotators to try to conclude about the class. If after the discussion, the annotator still doesn't agree with his colleagues, the \emph{gold class}\footnote{The class given to a text after the annotation process} is set as the majority class among the annotators but in the metadata of the text, it will be specified that the text was conflicting. Knowing if the text was conflicting will help us to perform the error analysis in the last parts of this report.
\\
\\
Here are some statistics about conflict annotations:
\begin{table}[h]
\centering
\begin{tabular}{|l|l|l|}
\hline
      & Non Conflict & Conflict \\ \hline
P1    & 393          & 131      \\ \hline
P2    & 346          & 120      \\ \hline
Total & 739          & 251      \\ \hline
\end{tabular}
\end{table}
\section{Feature Engineering}
Now that we have all our data annotated, we have to extract relevant information from it in order to perform the classification. The problem of identifying persuasion in a text is a relatively new question in NLP and we don't have any straightforward methodology to find relevant features. Thus, we'll implement the NLP standard features that are used widely in other sub-fields of language processing and then we'll use the state-of-the-art features discovered in Argumentation Mining. A lot of modules of DKPro helped us to create our features but we had to extend somehow the software for certain functionalities that were not in DKPro Core and TC (for example the \emph{Sentiment Analysis}).

\subsection{Lexical Features}   
The adjective lexical refers to the words and the vocabulary of a corpus. This part will deal with the extraction of meaningful features related to words, sentences, \glspl{token}\footnote{Tokenization is the process of breaking a stream of text up into words, phrases, symbols, or other meaningful elements called tokens.}, punctuations, ect...

\subsubsection{Tokens N-Grams}
In NLP, an n-gram is a contiguous sequence of n (with n integer) items from a given sequence of text or speech. As a result of, tokens n-grams are sequences of tokens from a text (\textit{Note: } For more information about units in linguistics such as words, tokens, \gls{lemma} and stemma, have a look at the glossary). The study of n-grams distribution in a corpus is an ancient technique \cite{tcjullmann77} in language processing and it's usage is common in the field.

In this study, we use a DKPro available feature that extracts the 10.000 most common 1,2 and 3-grams in all the corpus and returns for each text a 10000-dimensional binary vector. Each vector element corresponds to one of the 10000 extracted n-gram, its value is 1 if the text contains the n-gram, 0 otherwise.
\\
To better understand this feature, let's take a simple example. We consider that the set of 1-grams contains 10 elements such as follow: 

\begin{equation*}
E = \{2, 1990, a, born, dog, Frankfurt, in, I, was, zoo\}
\end{equation*}
\\

If the text input is:
\\
\centerline{\texttt{I was born in 1990}}
\\

DKPro will return the following binary 10-dimensional vector:
\begin{table}[h]
\center
\begin{tabular}{llllllllll}
2                       & 1990                   & a                      & born                   & dog                    & Frankfurt              & in                     & I                      & was                    & zoo                    \\ \hline
\multicolumn{1}{|l|}{0} & \multicolumn{1}{l|}{1} & \multicolumn{1}{l|}{0} & \multicolumn{1}{l|}{1} & \multicolumn{1}{l|}{0} & \multicolumn{1}{l|}{0} & \multicolumn{1}{l|}{1} & \multicolumn{1}{l|}{1} & \multicolumn{1}{l|}{1} & \multicolumn{1}{l|}{0} \\ \hline
\end{tabular}
\end{table}

As mentioned before, this feature is already implemented in DKPro and it's easy to use it in a pipeline.
\
\begin{figure}[h]
    \centering
    \includegraphics[width=1.05\textwidth]{fig/ngramfeature.png}
    \caption[Short caption]{NGram Feature in DKPro Java code}
    \label{fig:ngramfeature}
\end{figure}
\\
\\
As displayed on \ref{fig:ngramfeature}, the n-gram feature extractor takes three parameters: \texttt{PARAM\_NGRAM\_MIN\_N} for the minimal value of n, \texttt{PARAM\_NGRAM\_MAX\_N} for the maximal value of n and \texttt{PARAM\_NGRAM\_USE\_TOP\_K} which is the number of common n-grams retained.
Since we want the 10000 most common 1,2 and 3-grams, the parameters are set such as follow:

\begin{itemize}[label={}]
  \item \texttt{PARAM\_NGRAM\_MIN\_N = 1}
  \item \texttt{PARAM\_NGRAM\_MAX\_N = 3}
  \item \texttt{PARAM\_NGRAM\_USE\_TOP\_K = 10000}
\end{itemize}

We also implemented a subclass from the n-grams extractors to have the lemma instead of the tokens. We thought that it would give us better results since lemma refer to a general form of a word (ex: \emph{be} instead of \emph{are}, \emph{was}, \emph{is}, ect...) but in \emph{a posteriori} analyses, we don't see any improvements in our results.

The Tokens N-Grams feature will give us results for our \emph{Baseline Analysis}: more complex models and pipelines will systematically be compared to this ``simple" model.

\subsubsection{Tokens and Sentences}
We compute some statistics regarding tokens and sentences in a text:
\begin{itemize}
  \item Number of sentences and tokens in a text.
  \item Maximum size (in character) of a token and a sentence in a text.
  \item Minimum size (in character) of a token and a sentence in a text.
  \item Average size (in character) of tokens and sentences in a text. 
\end{itemize} 
\
As an example, if we consider the following text as input:
\\
\\
\fbox{\begin{minipage}{1\textwidth}
\textbf{208\_P2\_artcomment\_homeschooling.txt} ``Not having read any of the standard high school literature, people make references I don’t get." 
Got news for you. They're not making references to required high school readings. More likely Internet and pop culture. 
I hope you succeed getting some accountability in to the system. What is this issue, gun ownership? 
\end{minipage}}
\\
\\
The outputs are the following descriptive statistics:
\begin{itemize}
  \item \textbf{6} sentences in this text.
  \item The minimal sentence is ``Got news for you" and its size is \textbf{18} characters. 
  \item The maximal sentence is ``Not having read any of the standard high school literature, people make references I don’t get." and its size is \textbf{100} characters.
  \item The average size for a sentence is \textbf{53.3} characters.
\end{itemize}  
\
and besides:
\
\begin{itemize}
  \item \textbf{67} tokens.
  \item The minimal token is ``I" and its size is \textbf{1} characters. 
  \item The maximal sentence is ``accountability" and its size is \textbf{14} characters.
  \item The average size for a token is \textbf{4.0} characters.
\end{itemize}  

The features related to minimal sizes wouldn't give us much information about a text since a words like \emph{a} or \emph{I} are often used. On the other hand, the maximal size, total number and average size related features might be very useful since they quantify somehow the interest of an author in a conversation/debate.

\subsubsection{Other Lexical Statistics}
In the article \emph{Stance Classification of Ideological Debates}\cite{Hasan-Ng:IJCNLP:2013}, Hasan defines 3 simple features which help to perform stance classification:
\
\begin{itemize}
  \item Length in characters of a text.
  \item Ratio of tokens with more than 6 characters.
  \item Average number of tokens per sentence.
\end{itemize}  

\subsubsection{Punctuation Related Features}
First, we defined a set of features which simply compute the ratio per token of these 6 punctuation marks full stop, comma, question mark, exclamation mark, colon and quotation mark (\cref{punctuation}). This would tell us if the author is caring about his style of writing.
\\
\begin{figure}[h]
\begin{center}
\texttt{\textbf{.  ,  ?  !  :  "}}
\caption{\label{punctuation} 6 punctuation marks}
\end{center}
\end{figure}

Another punctuation feature inspired by \emph{An} and \emph{Walker} \cite{anand-EtAl:2011:WASSA2011} is the repeated punctuation feature that computes the number of repeated punctuation (such as ``!!!!!" or ``!??!") in a text. In practice, this feature requires the following \emph{regular expression}\footnote{In theoretical computer science and formal language theory, a regular expression (abbreviated regex or regexp) is a sequence of characters that forms a search pattern, mainly for use in pattern matching with strings, or string matching, i.e. "find and replace"-like operations.}: \textbf{\texttt{[?!.,]+}}
\
\begin{figure}[H]
    \centering
    \includegraphics[width=0.6\textwidth]{fig/multiplepunc.png}
    \caption[Short caption]{Piece of code for the multiple punctuation feature}
    \label{fig:multipunc}
\end{figure}
\
This feature can be a good representation of aggressiveness and poor argumentation in a debate, as it can be seen in the following example:
\\
\fbox{\begin{minipage}{1\textwidth}
\textbf{208\_P2\_artcomment\_homeschooling.txt} smarmy bastard 2011/09/19 at 6:00 PM "…ok so obviously , there are more free thinkers who would agree with you, .... and because there are more than one free thinker(s) , it becomes a group of free thinkers ...who all agree ... hmmm (head hits floor)" \_\_\_ 
Smarmy: now you're just being disagreeable. If many people independently think for themselves, without being told what, when, and how to think, it does not follow that they all think the same thing. "Following" is an attribute reserved for religion. 
Me thinks your head may have hit the floor too hard this time.
\end{minipage}}

\subsubsection{Multiple Capital Letters}
Another feature that can reflect the lack of seriousness is the number of words with multiple capital letters. In the following example, there are 3 words with multiple capital letters:
\\
\\
\fbox{\begin{minipage}{1\textwidth}
\textbf{3144\_P2\_artcomment\_public-private-schools.txt} WRONG - NO !! Perhaps bad psychology, bad child rearing, perhaps. They are paying for the public schools, its called  TAXES!! 
\end{minipage}} 	 	

\subsection{Part Of Speech Features}
In grammar, parts of speech (abbreviation: POS) are the linguistic categories of words such as verb, noun, ect... In DKPro the POS are modelled as subclasses of the class \emph{POS}, which is an annotation. 

\subsubsection{Ratio on common POS}
One simple feature, already implemented in DKPro\footnote{In TC Google code, have a look at de/tudarmstadt/ukp/dkpro/tc/features/syntax/POSRatioFeatureExtractor.java}, computes 11 ratios of 11 different over the total number of POS:

\begin{table}[h]
\center
\begin{tabular}{|l|l|l|}
\hline
\rowcolor[HTML]{9B9B9B} 
POS             & Abbreviation & Examples         \\ \hline
Adjective       & ADJ          & good, tall       \\ \hline
Adverb          & ADV          & quickly, lightly \\ \hline
Article         & ART          & a, the           \\ \hline
Cardinal Number & CARD         & one, eighty-two  \\ \hline
Conjunction     & CONJ         & for, and         \\ \hline
Noun            & N            & cat, Germany     \\ \hline
Exclamation     & O            & O, oh!           \\ \hline
Preposition     & PP           & above, within    \\ \hline
Pronoun         & PR           & I, she           \\ \hline
Punctuation     & PUNC         & ``.", ``;"       \\ \hline
Verb            & V            & to be, had       \\ \hline
\end{tabular}
\caption{\label{11pos} The 11 POS we consider for the Ratio POS feature}
\end{table}

The DKPro functions allow to compute the ratios and thus create the features very easily as seen on \cref{fig:11poscode}.
\
\begin{figure}[H]
    \centering
    \includegraphics[width=0.6\textwidth]{fig/11pos.png}
    \caption[Short caption]{Piece of code for the multiple punctuation feature}
    \label{fig:11poscode}
\end{figure}
\
\subsubsection{Comparative and Superlative}
The previous features don't consider the ratios of comparative and superlative (for adverbs and adjectives) in a text but they are relevant in debates since opponents usually keep on comparing the different point of views.

\subsubsection{Modal Verbs}
Again, in a more granulate analysis, we can evaluate the ratios for the 9 common ratios in English: can, could, may, might, must, shall, should, will and would.

\subsubsection{POS N-Grams}
By analogy with the Token NGrams feature, DKPro has a POS NGrams feature. As an example, if you consider the sentence:

\begin{table}[h]
\center
\begin{tabular}{cccccc}
This & Virginia & law & is & insane & .    \\
\texttt{ART}  & \texttt{NP}       & \texttt{NN}  & \texttt{V}  & \texttt{ADJ}    & \texttt{PUNC}
\end{tabular}
\end{table}\
\\
The POS 1-grams are: \texttt{ART, NP, NN, V, ADJ, PUNC}
\\
The POS 2-grams are: \texttt{ART\_NP, NP\_NN, NN\_V, V\_ADJ, ADJ\_PUNC}
\\
and so on \ldots
\\
Unfortunately, the POS n-grams tend to introduce some noise and redundancy in our classification, so we won't use them much.

\subsection{Syntactic features}
The \emph{syntax} is the study of how languages are constructed in a certain language. We'll define in this part the syntax related. Syntax is very descriptive, most of the syntactic representations of sentences, texts are often graphs, tables, ect... We'll see how we came with the quantitative features needed to perform the classification.

\subsubsection{Depth of the Dependency Tree}
\textbf{Dependency Tree}
\\
\\
In V {\'A}gel's works \cite{ágel2006dependency}, the dependency tree is a graph that maps the relations between the different grammar units in a sentence. The theory behind is long and arduous and we leave to the reader the study of dependency trees. Nevertheless, we give in the following figure (\ref{fig:deptree}), a simple example of a dependency tree.

For the sentence \texttt{I shot an elephant in my pajamas}, we get the following tree:
\
\begin{figure}[H]
    \centering
    \includegraphics[width=0.7\textwidth]{fig/deptree.png}
    \caption[Short caption]{A dependency tree}
    \label{fig:deptree}
\end{figure}
\
To quantify how complex a sentence can be, we took inspiration on Christian Stab work on Argumentative Discourse \cite{TUD-CS-2014-0882} by calculating the depth of the dependency tree for every sentence. The depth of a tree is the number of edges between the first node and the furthest extremity in the tree. In \cref{fig:deptree}, the depth of the tree is 5.
\
\begin{figure}[H]
    \centering
    \includegraphics[width=0.7\textwidth]{fig/deptreepath.png}
    \caption[Short caption]{5 edges between the summit S and the extremities Det and N}
    \label{fig:deptreepath}
\end{figure}
\
\\
\\
\textbf{Building features with this metric}
\\
The dependency tree is available on DKPro with the \emph{MaltParser}\cite{Nivre_aquick}. For every input sentence, it returns the corresponding dependency tree, such as follow: 
\begin{figure}[h]
\begin{center}
\texttt{(S (NP I) (VP (V shot) (NP (Det an) (N elephant) (PP (P in) (NP (Det my) (N pajamas))))))}
\caption{\label{dkprotree} MaltParser's output tree}
\end{center}
\end{figure}
Certain functions allow to evaluate the depth of this kind of tree. Even so, our base unit for the classification is the text (and not the sentence), so we need to compute certain statistics over the sentences:
\\
\textbf{Maximal Tree}

If \emph{depth} is a function that returns the depth of a tree, the biggest tree in a text has a size of:
\begin{equation*}
\underset{s \in text}{max} \; depth(s)
\end{equation*}
\\
Where the dummy variable \emph{s} corresponds here to a sentence.
\\
\\
\textbf{Average length of a tree}
\\
\\
Here we simply calculate the tree depth average on all the sentences:
\begin{equation*}
\frac{1}{|s|} \; \displaystyle\sum_{s \in text} depth(s)
\end{equation*}

\subsubsection{Dependency Rules}
Similarly to the tokens n-grams and the POS n-grams, it's possible to define dependencies n-grams, simply called dependency rules \cite{TUD-CS-2014-0882}. As an example, some of the dependency rules from the previous tree (\cref{fig:deptree}) are : VP $\rightarrow$ NP, NP $\rightarrow$ PP $\rightarrow$ NP, ect...

In our study, we extract 5000 dependency rules from the all corpus and compute binary vectors that show the presence or not of a dependency rule.

\subsubsection{Subordinate clauses}
\textbf{Clause}
\\
In grammar, a clause is the smallest grammatical unit that can express a complete proposition. 
\\
\textbf{Subordinate clause}
\\
Subordination as a concept of syntactic organization is associated closely with the distinction between coordinate and subordinate clauses. One clause is subordinate to another, if it depends on it. The dependent clause is called a subordinate clause and the independent clause is called the main clause.

We can distinguish 5 kind of subordinate clauses:
\begin{itemize}
  \item \texttt{Clause S} - simple declarative clause, i.e. one that is not introduced by a (possible empty) subordinating conjunction or a \emph{wh-word}\footnote{interrogative word or question word} and that does not exhibit subject-verb inversion.
  \item \texttt{Clause SBAR} - Clause introduced by a (possibly empty) subordinating conjunction.
  \item \texttt{Clause SBARQ} - Direct question introduced by a wh-word or a wh-phrase. Indirect questions and relative clauses should be bracketed as SBAR, not SBARQ.
  \item \texttt{Clause SINV} - Inverted declarative sentence, i.e. one in which the subject follows the tensed verb or modal. 
  \item \texttt{Clause SQ} - Inverted yes/no question, or main clause of a wh-question, following the wh-phrase in SBARQ.
\end{itemize} 
To evaluate the importance of those clauses in the text, we built a feature that calculate the maximum number of clauses per sentence. This feature was also inspired by Stab work \cite{TUD-CS-2014-0882}.
\
\begin{figure}[H]
    \centering
    \includegraphics[width=0.7\textwidth]{fig/subclause.png}
    \caption[Short caption]{Clause Ratio Feature}
    \label{fig:subclause}
\end{figure}
\

\subsection{Sentiment Analysis Feature}
Sentiment Analysis, or Opinion Mining refer to NLP techniques of detecting subjective information out of textual data. The research work related to the field really exploded over the past decade, especially when social media and its corollary, the availability of high subjective and sentimental data, emerged. In this work, we use the standard state-of-the-art tool for researchers which is \emph{GPL Stanford Deep Learning for Sentiment Analysis}\footnote{http://nlp.stanford.edu/sentiment/code.html}. 

This tool assign to each sentence 5 percentage coefficients labelled \texttt{Very Negative}, \texttt{Negative}, \texttt{Neutral}, \texttt{Positive} and \texttt{Very Positive}. Those coefficients are calculated by recursive deep models that are detailed in Socher and Perelygin article \cite{Socher_recursivedeep}.

\
\begin{figure}[H]
    \centering
    \includegraphics[width=0.8\textwidth]{fig/sentimenttree.png}
    \includegraphics[width=0.4\textwidth]{fig/sentimentcoeff.png}
    \caption[Short caption]{ Recursive Neural Tensor Network and the resulting sentiment coefficients}
    \label{fig:sentcoeff}
\end{figure}

Stanford's Sentiment Analysis tool was not available in DKPro, and we had to partially\footnote{The two software don't work on the same annotations} integrate it in the pipeline.

\subsubsection{Sentiment Coefficients}
We call \emph{sentiment coefficients} the 5 output coefficients returned by Stanford's Sentiment Analysis tool. Again, those coefficients are calculated on the sentences and thus, we have to perform a statistical analysis on the sentences. 

We denote the 5 sentiment coefficients with symbols as follows: \texttt{(-- - 0 + ++)} and $f_{c}$ is the function that given one sentiment c returns the corresponding coefficients in the sentence. Thus, we compute the minimum, the maximum, the average and the standard deviation of those 5 coefficients which gives us 20 metrics to evaluate the sentiment distribution in our text:
\begin{equation*}
\forall c \in \texttt{\{--, -, 0, +, ++\}}, \: \:
\begin{cases}
min_c = \underset{s \in text}{min} \: \: f_c(s) \\
max_c = \underset{s \in text}{max} \: \: f_c(s) \\
\mu_c = \frac{1}{|s|} \; \displaystyle\sum_{s \in text} f_c(s) \\
\sigma_c = \sqrt{\frac{\displaystyle\sum_{s \in text} (f_c(s) - \mu_c)^2}{|s|}}
\end{cases}
\end{equation*} 

\subsubsection{Sentiment Fluctuation}
The sentiments may vary slightly or significantly from a sentence to another. We can then define the \emph{sentiment rules} which model the transition from one state to another. Since we have 5 type of coefficients, it results in 25 rules (ex: -- $\rightarrow$ +, 0 $\rightarrow$ ++, - $\rightarrow$ -). 

In comparison with token n-grams and dependency rules, 25-dimensional binary vectors are built and represent the absence and the presence of a rule. If we consider as input the two following sentences:

\texttt{It is a choice.} 

\texttt{Independent choice is what makes American values so precious.}
\\
The corresponding sentiment fluctuation after using the sentiment analysis tool is: 0 $\rightarrow$ +
\\

\subsection{LDA}
The Latent Dirichlet Allocation is a generative model\footnote{A generative model is a model for randomly generating observable data, typically given some hidden parameters.} that allows sets of observations to be explained by unobserved groups that explain why some parts of the data are similar. Intuitively, it’s a way of automatically discovering topics that these sentences contain. 
\\
Suppose you have the following set of sentences:
\\

1) I like to eat mango and apples.

2) I ate a mango and carrot smoothie for breakfast.

3) Horses and cats are cute.

4) My sister adopted a cat yesterday.

5) Look at this cute hamster munching on a piece of carrot.
\\
\\
Intuitively, The LDA will produce those rules:
\begin{itemize}
  \item Sentences 1 and 2: 100\% Topic $\alpha$
  \item Sentences 3 and 4: 100\% Topic $\beta$
  \item Sentence 5: 60\% Topic $\alpha$, 40\% Topic $\beta$
  \item Topic $\alpha$: 30\% mango, 15\% apple, 10\% breakfast, 10\% munching... (food and eating)
  \item Topic $\beta$: 20\% cat, 20\% horses, 20\% cute, 15\% hamster... (animals)
\end{itemize} 

LDA performs its discovery by representing documents as mixture of topics  that spit out words with certain probabilities. In this section, we introduce LDA in a intuitive way but please have a look at Blei, Jordan and Ng work\cite{Blei:2003:LDA:944919.944937} on it.

\section{The Classifier and Performances}
In Data Mining, once we have defined all the features that describe our data, we need to train a model on those features and evaluate the performances of the model. Even if the results can vary a lot from one classifier to another, in this study we're more interested in how perform the feature rather than how perform a certain classifier. Thus, we'll use a common state-of-the-art classifier called \emph{Support Vector Machine} or \emph{SVM}.

\subsection{SVM Classifier}
In the field of machine learning, SVMs are supervised learning models that perform classification (also regression, but it's not the purpose of our study) by finding the hyperplane that maximizes the margin between the two classes. The vectors that define the hyperplane are called \emph{support vectors} \cite{Boser:1992}.

The main advantage of SVM is that, as we'll see later, it can separate non linearly separable data thank to its \emph{kernel} by adding dimensions and transform the data into linearly separable data, as it can be seen on the following figures:

\
\begin{figure}[H]
    \centering
    \includegraphics[width=0.8\textwidth]{fig/svm-nonlin.png}
    \caption[Short caption]{In this 2D representation, the blue and red classes are not linearly separable}
    \label{fig:svm-nonlin}
\end{figure}

\
\begin{figure}[H]
    \centering
    \includegraphics[width=0.8\textwidth]{fig/svm-lin.png}
    \caption[Short caption]{By transforming the data and adding one dimension, the classes are linearly separable}
    \label{fig:svm-lin}
\end{figure}

We shortly outline the different steps of the SVM algorithm:
\\
\\
\textbf{Algorithm}
\begin{itemize}
  \item Define an optimal hyperplane: maximize margin
  \item Extend the above definition for non-linearly separable problems: have a penalty term for misclassifications.
  \item Map data to high dimensional space where it is easier to classify with linear decision surfaces: reformulate problem so that data is mapped implicitly to this space.
\end{itemize} 
\\
To specify in more details, here is how the optimal hyperplane and the margin are defined:
\\
\begin{figure}[H]
    \centering
    \includegraphics[width=0.8\textwidth]{fig/hypopt1.png}
    \includegraphics[width=0.8\textwidth]{fig/hypopt2.png}
    \caption[Short caption]{Margin and Hyperplane Optimization for SVM}
    \label{fig:hypopt}
\end{figure}
\\
The geometric parameters \emph{b} and \emph{w} are found using \emph{Quadratic Programming}\footnote{Quadratic programming (QP) is a special type of mathematical optimization problem. It is the problem of optimizing (minimizing or maximizing) a quadratic function of several variables subject to linear constraints on these variables.} on this following function:
\begin{equation*}
min \; \frac{1}{2} ||w||^2 
\end{equation*}
\begin{equation*}
s.t. y_i \; (w \; x_i + b) \geq 1, \forall x_i
\end{equation*}
If the data is linearly separable, the solver is supposed to find a unique minimum. Ideally SVM analysis should create an hyperplane that completely separates the classes into two non-overlapping groups. However, in practice, perfect separation may not be possible, or it may result in a model with so many cases that the model does not classify correctly. In this situation SVM finds the hyperplane that maximizes the margin and minimizes the misclassifications. Thus the slack variable is introduced: It correspond to the ratio of variables that are allowed to fall off the margin.

\
\begin{figure}[H]
    \centering
    \includegraphics[width=0.8\textwidth]{fig/svm-slack.png}
    \caption[Short caption]{Slack variable: In this example, two instances are misclassified}
    \label{fig:svm-slack}
\end{figure}

SVM tries to maintain the slack variable as close to zero as possible while maximizing margin. However, it does not minimize the number of misclassifications (In computational complexity theory, NP-complete problem\footnote{NP-complete problems are in NP, the set of all decision problems whose solutions can be verified in polynomial time}) but the sum of distances from the margin hyperplanes. Besides, with the introduction of the new variables, the objective function and the constraints change:

\
\begin{figure}[H]
    \centering
    \includegraphics[width=0.8\textwidth]{fig/svm-C.png}
    \caption[Short caption]{New optimization problem}
    \label{fig:svm-C}
\end{figure}

The simplest way to separate two groups of data is with a straight line (1 dimension), flat plane (2 dimensions) or an N-dimensional hyperplane. However, there are situations where a non-linear region can separate the groups more efficiently. SVM handles this by using a kernel function (non-linear) to map the data into a different space where a hyperplane (linear) cannot be used to do the separation. It means a non-linear function is learned by a linear learning machine in a high-dimensional feature space while the capacity of the system is controlled by a parameter that does not depend on the dimensionality of the space. This is called kernel trick which means the kernel function transform the data into a higher dimensional feature space to make it possible to perform the linear separation.  

\
\begin{figure}[H]
    \centering
    \includegraphics[width=0.8\textwidth]{fig/svm-kernel.png}
    \caption[Short caption]{Kernel Function: Non-linear separation}
    \label{fig:svm-kernel}
\end{figure}

The two kernel functions used in practice are the following:
\\
\textbf{Polynomial Kernel}
\begin{equation*}
\forall x_i, x_j \: \: instances, \: \: k(x_i, x_j) = (x_i.x_j)^d, d \in \mathbb{N}
\end{equation*}
\\
\textbf{Gaussian Kernel}
\begin{equation*}
\forall x_i, x_j \: \: instances, \: \: k(x_i, x_j) = exp(-\frac{||x_i - x_j||^2}{2 \sigma^2})
\end{equation*}

\subsection{Weka's SMO Classifier}
The machine learning tools available on DKPro TC are based on \emph{Weka} components. The implementation we'll use for SVM is John C. Platt's \emph{Sequential Minimal Optimization} (SMO) algorithm\cite{Platt:1999}.  This implementation globally replaces all missing values and transforms nominal attributes into binary ones. It also normalizes all attributes by default. We'll use the polynomial kernel in our experiments since it was \emph{a posteriori} the most efficient one.

\subsection{Evaluate the performances of our classification}
To evaluate the performances of our classification, we should define a bunch of metrics that quantify the prediction capacities of our system (features + algorithm). We'll first see the different scenarios of prediction and then the measures we use to quantify the performances of our prediction. With the type of data we have, we can define 4 types of scenarios. If certain terms are unknown for you, please have a look at \cref{sec:bincla} on Binary Classification.
\\

\textbf{The In-Domain Cross-Validation}
We perform a 10 folds cross-validation (\cref{sec:bincla}) for all the data in one of the six domains we previously defined. At the result of, for every experiment we run, we get 6 in-domain cross-validation. The reported results are the \emph{Accuracy} and teh \emph{Macro F-measure} as defined in \cref{sec:bincla}.
\\
For the domain homeschooling, we'll perform a \emph{leave-one-out} cross-validation\footnote{Leave-one-out cross-validation (LOOCV) is a particular case of n-fold cross-validation with n = number of instances. Since we have 29 text data for homeschooling, we perform a 29 folds cross-validation.}
\\

\textbf{The Full Cross-Validation}
We perform a 10 folds cross-validation\ref{sec:bincla} for all the data in one of the six domains we previously defined. At the result of, for every experiment we run, we get 6 in-domain cross-validation.
\\

\textbf{The Cross-Domain Validation}
For each domain, the data related to the domain is considered as a test set and the rest of the data (5 other domains) forms the test set. This evaluation scenario is somehow the most important since it reveals how \emph{general} are the features and how well they can predict the persuasiveness in general.
\\

\textbf{The Full Cross-Domain Validation}
We aggregate the results of the 6 cross-domain validations such as follow:
\\
If $Dom$ is a domain in D = \{homeschooling, redshirting, prayer-in-schools, public-private-schools, mainstreaming, single-sex-education\} and $TP_{Dom}$, $TN_{Dom}$, $FP_{Dom}$ and $FN_{Dom}$ are the statistical measures for the performances of the binary classification (\cref{sec:bincla}), the confusion matrix of the Full Cross-Domain Validation scenario is defined as followed:
\\
\begin{table}[h]
\center
\begin{tabular}{|c|l|l|}
\hline
\multicolumn{1}{|l|}{} & \multicolumn{2}{l|}{Prediction} \\ \hline
\multirow{2}{*}{\begin{tabular}[c]{@{}c@{}}Actual\\ Value\end{tabular}} & $TP_{FCD}$ & $FP_{FCD}$ \\ \cline{2-3} 
 & $FN_{FCD}$ & $TN_{FCD}$ \\ \hline
\end{tabular}
\end{table}
\\
With:
\begin{equation*}
TP_{FCD} = \sum\nolimits_{Dom \in D} TP_{Dom}
\end{equation*}
\begin{equation*}
TN_{FCD} = \sum\nolimits_{Dom \in D} TN_{Dom}
\end{equation*}
\begin{equation*}
FP_{FCD} = \sum\nolimits_{Dom \in D} FP_{Dom}
\end{equation*}
\begin{equation*}
FN_{FCD} = \sum\nolimits_{Dom \in D} FN_{Dom}
\end{equation*}

The \emph{Accuracy} and the \emph{F-measure} for this scenario directly come from the previous measures.